\documentclass[xcolor=x11names, aspectratio=169,usenames,dvipsnames]{beamer}
\usepackage[british]{babel}
\usepackage{amsmath}
\usepackage{amssymb}
\usepackage{amsfonts}
\usepackage{mathpazo}
\usepackage{enumerate}
\usepackage{array,booktabs}
\usepackage{tikz}
\usepackage{mathdots}
\usepackage{verbatim}
\usepackage{multirow}
\usepackage{tabularx}
\usetikzlibrary{shadows,matrix,backgrounds,patterns,arrows,decorations.markings,shapes,positioning,calc,chains,scopes,fit}
\usepackage{caption}

\usetheme[titleformat title=regular,titleformat frame=regular,titleformat section=allcaps,numbering=fraction]{metropolis}

\author[T.\ Evans \&\ F.\ Kußmaul]{\large Tim Evans\inst{1}, \and Felix Kußmaul\inst{2}}
\title[Mining Paper Catalogues]{\Large Mining Paper Catalogues}
\subtitle{\normalsize A Multilingual Solution to Reduce Verbose Fields to Consistent Terminology}
\institute[York, Cologne]{\inst{1} Archaeology Data Service, University of York \and \inst{2} Archaeological Institute, University of Cologne}
\date[31 August 2017]{\vspace*{1em}23rd Annual Meeting EAA, Maastricht\\[.5em] 31 August 2017\vspace*{1em}}

\titlegraphic{
    \tikz[overlay,remember picture]
        \node[xshift=-5em,yshift=3em,at=(current page.south east), anchor=south east] {
            \includegraphics[width=0.25\textwidth]{img/archaide.eps}
        };
}
\input{config.tex}
\definecolor{morange}{HTML}{FF8200}
\metroset{block=fill}
 
\begin{document}

\begin{frame}[plain]
\titlepage
\end{frame}

\section{Motivation}

\begin{frame}{Data Source}
\begin{center}
\begin{figure}
\includegraphics[width=.875\paperwidth]{img/consp.jpg}
\caption{Sample from \cite{consp}.}
\end{figure}
\end{center}
\end{frame}

\begin{frame}{Oh dear!}
\begin{block}{Problem}\vspace{.5em}
Running texts contain a lot of \emph{irrelevant information} (for machine processing).

This makes database lookups without keywords \alert{\textbf{extremely inefficient}}.
\end{block}
\end{frame}

\begin{frame}[fragile]{}
\begin{minipage}[t]{0.45\textwidth}
What we \textbf{have}:\medskip

\begin{figure}
\includegraphics[width=1.0\textwidth]{img/consp_ex.jpg}
\end{figure}
\end{minipage}\hfill\pause
\begin{minipage}[t]{0.45\textwidth}
What we \textbf{want}:\medskip
{\scriptsize
\begin{verbatim}
{
   "form": "23.1",
   "origin": "Italy",
   "decoration": "none",
   "occurs": "uncommon"
},
{
   "form": "23.2",
   "origin": "Italy, not Padana",
   "occurs": "Mediterranean region;
              North-Italy"
}
\end{verbatim}
}
\end{minipage}\pause\medskip

\begin{minipage}[t]{0.45\textwidth}
\begin{center}
\alert{\textbf{UNSTRUCTURED DATA}}
\end{center}
\end{minipage}\hfill
\begin{minipage}[t]{0.45\textwidth}
\begin{center}
\alert{\textbf{STRUCTURED DATA}}
\end{center}
\end{minipage}
\end{frame}

\section{Text Mining}%\printSectionYes

\begin{frame}{Classification}
\begin{block}{Definition: Text Mining}\vspace{.5em}
\textbf{Text Mining} is a \alert{general term} covering several different ideas\pause, e.\,g.:
\begin{itemize}[<+->]
\item Information retrieval
\item Statistical analysis
\item \textbf{Information extraction}
\item\dots
\end{itemize}
\end{block}
\end{frame}

\begin{frame}{Information Extraction}
\begin{block}{Definition: Information Extraction (IE)}\vspace{.5em}{
\enquote{\textit{[\textbf{IE} refers to] the \textbf{identification} and extraction of instances of a particular class of events or relationships in a \textbf{natural language text} and their \textbf{transformation} into a structured representation.}} \hfill -- Grishman 1997, Eikvil 1999}
\end{block}\pause\bigskip

\large{Some other facts about Information Extraction:}\normalsize
\begin{itemize}[<+->]
\item Computer scientists have a hard time with IE (for over \alert{30 years} now!)
\item IE is \textbf{\alert{really super difficult}} and \alert{\textbf{often inaccurate}}.
\end{itemize}
\end{frame}

%\begin{frame}{Why underestimation is bad}
%\begin{center}
%\begin{figure}
%\includegraphics[width=\textwidth]{img/xkcd.png}
%\caption{I can relate to this. [Source: xkcd.com/1831]}
%\end{figure}
%\end{center}
%\end{frame}

\begin{frame}{Sorry!}
\begin{large}
\begin{alertblock}{\large DISCLAIMER}\vspace{.5em}
In this presentation, we show \textbf{preliminary} results, as this project is still work in progress.
\end{alertblock}
\end{large}
\end{frame}

\begin{frame}{IE Process Pipeline with UIMA}
\tikzset{%
  materia/.style={draw, text centered, minimum height=1.8em, minimum width=5em, font=\footnotesize},
  etape/.style={materia, fill=blue!20, rounded corners},
  doc/.style={materia, fill=green!20,shape=document, text width=6em, inner ysep=5pt},
  lab/.style={anchor=base,text width=5em,font=\bfseries\itshape\scriptsize},
  back group/.style={fill=yellow!20,rounded corners, draw=black!50, thick, inner ysep=5pt,minimum height=2cm},
  imp/.style={etape}
}
\tikzstyle{myarrows}=[-stealth',line width=.8mm]
\tikzstyle{doublea}=[stealth'-stealth',line width=.3mm]

\tikzset{%
  cascaded/.style = {%
    general shadow = {%
      etape,
      shadow scale = 1,
      shadow xshift = .8ex,
      shadow yshift = .8ex,
      draw},
    general shadow = {%
      etape,
      shadow scale = 1,
      shadow xshift = .4ex,
      shadow yshift = .4ex,
      draw},
    draw}}
    
\pgfdeclarelayer{background}
\pgfdeclarelayer{foreground}
\pgfsetlayers{background,foreground,main}
 
\begin{figure}
\resizebox{\textwidth}{!}{
\begin{tikzpicture}[node distance=1.5em and 3em, align=center]
\node[etape] (token) {Stanford};
\node[etape, right=of token] (lemma) {ClearNLP};
\node[etape, right=of lemma] (pos) {OpenNLP};
\node[etape, right=of pos] (ner) {CoreNLP};
\node[etape, right=of ner] (ie) {UIMA Ruta};

\node[above=of token,lab] (t) {Tokenisation};
\node[above=of lemma,lab] (l) {Lemmatisation};
\node[above=of pos,lab] (p) {POS-Tagging};
\node[above=of ner,lab] (n) {NER};
\node[above=of ie,lab] (i) {Information Extraction};

\begin{pgfonlayer}{foreground}
\node[back group] (start) [fit=(token) (t)]{};
\node[back group] (1) [fit=(lemma) (l)]{};
\node[back group] (2) [fit=(pos) (p)]{};
\node[back group] (3) [fit=(ner) (n)]{};
\node[back group] (end) [fit=(ie) (i)]{};
\end{pgfonlayer}

\node[doc, above=3.5em of start] (doc) {unstructured document};
\node[doc, above=3.5em of end,node distance=2em] (struc) {structured data};

\node[etape, cascaded, below=3em of token] (st) {PosMapper};
\node[etape, cascaded, below=3em of lemma] (sl) {CoreNLP};
\node[etape, cascaded, below=3em of pos] (sp) {MatePos};
\node[etape, cascaded, below=3em of ner] (sn) {OpenNLP};
\node[etape, cascaded, below=3em of ie] (si) {CoreNLP};

\draw[doublea] (token) edge (st);
\draw[doublea] (lemma) edge (sl);
\draw[doublea] (ner) edge (sn);
\draw[doublea] (pos) edge (sp);
\draw[doublea] (ie) edge (si);

\node[lab,above=0.4em of 3,xshift=-1.6em,align=right,font=\upshape\footnotesize,text width=9em](uima) {UIMA implementation};

\begin{pgfonlayer}{background}
\node[dashed,back group, fill=orange!20] [fit=(uima) (start) (1) (2) (3)] {};
\end{pgfonlayer}

\draw[myarrows] (doc) edge (start) (start) edge (1) (1) edge (2) (2) edge (3) (3) edge (end) (end) edge (struc);
\end{tikzpicture}
}
\caption{IE Process Pipeline.}
\end{figure}
\end{frame}

\begin{frame}{POS-Tagging}
\begin{figure}
\tikzstyle{block} = [rectangle, draw, fill=blue!10, rounded corners, text centered, minimum width=1.5em,font=\footnotesize\ttfamily, node distance=2em]
\begin{tikzpicture}[node distance = .5em, font=\bfseries\large,baseline, text height=1.5ex,text depth=.25ex,highlight/.style={text=orange}]
\node (a) at (0,0) {The};
\node[right=of a,onslide={<2-> highlight}] (b) {quick};
\node[right=of b,onslide={<2-> highlight}] (c) {brown};
\node[right=of c,onslide={<2-> highlight}] (d) {fox};
\node[right=of d] (e) {jump};
\node[right=of e] (f) {over};
\node[right=of f] (g) {the};
\node[right=of g,onslide={<2-> highlight}] (h) {lazy};
\node[right=of h,onslide={<2-> highlight}] (i) {dog};
\node[right=of i] (j) {.};

\node[block,fill=Fuchsia!30,below of=a] {DT};
\node[block,fill=yellow!30,below of=b] {JJ};
\node[block,fill=yellow!30,below of=c] {JJ};
\node[block,fill=RoyalBlue!30,below of=d] {NN};
\node[block,fill=green!30,below of=e] {VBD};
\node[block,fill=orange!30,below of=f] {IN};
\node[block,fill=Fuchsia!30,below of=g] {DT};
\node[block,fill=yellow!30,below of=h] {JJ};
\node[block,fill=RoyalBlue!30,below of=i] {NN};
\node[block,fill=gray!30,below of=j] {.};

\node at (0,-1.5) {};
\node[above of=e,gray,node distance=1.3em,font=\mdseries] {\footnotesize jumps}; 
\end{tikzpicture}
\caption{POS-tagging examples after lemmatisation.}
\end{figure}
\end{frame}

\begin{frame}[fragile]{Adapting the NER}\large
Most NERs (e.\,g.\ \textbf{Stanford CoreNLP}) only recognise 8 entities types:

\begin{center}\ttfamily
\begin{tabular}{lp{4em}l}
PERSON&&DATE\\
ORGANIZATION&&TIME\\
LOCATION&&MONEY\\
PERCENT&&MISC
\end{tabular}
\end{center}

So we have to add the \alert{custom entity type \texttt{FORM}}.
\end{frame}

\begin{frame}{Two approaches for NER}
\begin{minipage}[t]{0.45\textwidth}
\textbf{Rule-based approach}
\begin{itemize}
\item High precision, but lower recall

$\Rightarrow$ \alert{Many many rules?!}
\end{itemize}
\visible<2->{
\begin{figure}
\includegraphics[width=\textwidth]{img/eleph.jpg}
\caption{Excerpt from \cite{eleph}.}
\end{figure}}
\end{minipage}\pause\pause\hfill
\begin{minipage}[t]{0.45\textwidth}
\textbf{Machine-learning approach}
\begin{itemize}
\item Lower precision, but high recall
\item \alert{Needs to be trained!}
\end{itemize}%\vspace*{-1em}
\visible<4>{
\begin{figure}
\hspace*{-1em}\includegraphics[width=1.05\textwidth]{img/iepy.png}
\caption{Manually annotated sentence from \cite{consp} in \texttt{iepy}.}
\end{figure}
}
\end{minipage}
\end{frame}

\begin{frame}{Temporal Expressions}
With \textbf{\textit{\textsc{HeidelTime}}} temporal expressions are mapped to TIMEX3 standard
\begin{center}
{\renewcommand{\arraystretch}{1.2}%
\begin{tabular}{rcl}
\texttt{around 140 B.\,C.}&$\longmapsto$&\texttt{APPROX BC0140}\\
\texttt{Spätes 3.–4.\ Jh.\ n.\,Chr.}&$\longmapsto$&\texttt{END 02};~~ \texttt{03}\\
\texttt{second quarter first century B.\,C.}&$\longmapsto$&\texttt{XXXX-Q2 BC00}\\
\texttt{first half third century A.\,D.}&$\longmapsto$&\texttt{XXXX-H1 02}\\
\end{tabular}
}
\end{center}

\textbf{\textsc{HeidelTime}} supports many other languages, e.\,g.\ German, Italian, French, \dots
\end{frame}

\begin{frame}[fragile]{Relation Extraction}
\begin{center}
\begin{tabularx}{\textwidth}{CCC}
\toprule
\textbf{Subject}&\textbf{Relation}&\textbf{Object}\\\midrule
quick brown fox&jump over&lazy dog\\
K 612&dates&\texttt{03}\footnote{\enquote{4th century A.\,D.}}\\
Form 23&dates&\texttt{XXXX-Q2 00}\footnote{\enquote{second and third quarters of the first century A.\,D.}}\\
Subform 23.2&dates&\texttt{XXXX-Q2 00}~~\\
\bottomrule
\end{tabularx}\bigskip\pause

\begin{minipage}{0.3\textwidth}\flushright
\textbf{$\boldsymbol{\Rightarrow}$ e.\,g.}
\end{minipage}\hfill
\begin{minipage}{0.65\textwidth}
{
\begin{verbatim}
{  "form":   "23.2",
   "dating": "XXXX-Q2 00"  }
\end{verbatim}
}
\end{minipage}
\end{center}
\end{frame}

%NOTE: supports Pleiades and GeoNames

\begin{frame}{Locations with \textsc{HeidelPlace}}
\begin{figure}
\hspace*{-1em}\includegraphics[width=1.025\textwidth]{img/heidelplace.png}
\caption{Screenshot of \textsc{HeidelPlace}.}
\end{figure}
\end{frame}

\section{Multilingualism}

\begin{frame}{Background}
Two problems:
\begin{itemize}
\item Linguistic
\item Conceptual
\end{itemize}
\end{frame}

\begin{frame}{Different languages}
\begin{center}
\begin{figure}
\includegraphics[width=\textwidth]{img/tim_vocab_1.png}
\end{figure}
\end{center}
\end{frame}

\begin{frame}{Different traditions}
\begin{center}
\begin{figure}
\includegraphics[width=\textwidth]{img/tim_plate_platter_dish.jpg}
\caption{Plate, platter or dish?}
\end{figure}
\end{center}
\end{frame}

\begin{frame}{Creating controlled vocabularies}
\begin{itemize}
\item Sherd type (e.g. rim)
\item Form (e.g. plate)
\item Decoration form (e.g. burnished)
\item Decoration color (e.g. yellow)
\item Fabric (e.g. bla)
\end{itemize}
\end{frame}

\begin{frame}{Lessons from ARAIDNE}
Using tools developed for the ARIADNE project by the Hypermedia Research Group at the University of South Wales \hfill\raisebox{-.9em}{\includegraphics[width=4cm]{img/tim_ariadne_logo.png}}\newline
Creation of a neutral spine based on the Getty Institute's Art and Architecture Thesaurus (AAT)\newline
more bla\newline
more bla\newline
more bla\newline
\end{frame}

\section{Outlook}

\begin{frame}{Challenges to meet}
challenges:

choice of tools, coreferences in text, eloquence of archaeologists, maybe calculating F-value?

\textsc{HeidelTime}:\vspace{-1em}
\begin{center}
second and \textbf{third quarter} of the \textbf{first century A.\,D.}\quad$\longmapsto$\quad\texttt{XXXX-Q3};~~\texttt{00}
\end{center}
\end{frame}

\begin{frame}{References}

\printbibliography[heading=none]
\end{frame}


\begin{frame}[plain]
\vfill\vfill\vfill
\begin{center}\Large
Thank you very much for your attention!\\\bigskip

Questions?
\end{center}\vfill\vfill

\hfill
\begin{minipage}{0.7\textwidth}\scriptsize
\begin{flushright}
This project has received funding from the European Union's Horizon 2020 research and innovation programme under grant agreement \textnumero\ 693548
\end{flushright}
\end{minipage}\hspace*{1em}
\begin{minipage}{0.1\textwidth}
\includegraphics[width=\textwidth]{img/eu_flag.ps}
\end{minipage}
\end{frame}

\maketitle

\end{document}
