\documentclass[xcolor=x11names, aspectratio=169,usenames,dvipsnames]{beamer}
\usepackage[british]{babel}
\usepackage{amsmath}
\usepackage{amssymb}
\usepackage{amsfonts}
\usepackage{mathpazo}
\usepackage{enumerate}
\usepackage{array,booktabs}
\usepackage{tikz}
\usepackage{mathdots}
\usepackage{verbatim}
\usepackage{multirow}
\usepackage{tabularx}
\usetikzlibrary{matrix,backgrounds,patterns,arrows,decorations.markings,shapes,positioning}
\usepackage{caption}

\usetheme[titleformat title=regular,titleformat frame=regular,titleformat section=allcaps,numbering=fraction]{metropolis}

\author[F.\ Kußmaul \&\ T.\ Evans]{{\Large Felix Kußmaul}\hfill\raisebox{-.9em}{\includegraphics[width=4cm]{img/uco.jpg}}\\[1.2em]{\Large Dr Tim Evans}\hfill\raisebox{-.7em}{\includegraphics[width=4cm]{img/york.jpg}}}
\title[Mining Paper Catalogues]{\Large Mining Paper Catalogues}
\subtitle{\normalsize A Multilingual Solution to Reduce Verbose Fields to Consistent Terminology}
\institute[Cologne, York]{EAA Maastricht 2017}
\date[31 August 2017]{\ \\[.5em]31 August 2017}

\input{config.tex}
\definecolor{morange}{HTML}{FF8200}
\metroset{block=fill}
 
\begin{document}

\maketitle

\section{Motivation}

\begin{frame}{Data Source}
\begin{center}
\begin{figure}
\includegraphics[width=.875\paperwidth]{img/consp.jpg}
\caption{Sample from \emph{Conspectus} catalogue.}
\end{figure}
\end{center}
\end{frame}

\begin{frame}{Oh dear!}
\begin{block}{Problem}\vspace{.5em}
Running texts contain a lot of \emph{irrelevant information} (for machine processing).

This makes database lookups \alert{\textbf{extremely inefficient}}.
\end{block}
\end{frame}

\begin{frame}[fragile]{}
\begin{minipage}[t]{0.45\textwidth}
What we \textbf{have}:\medskip

\begin{figure}
\includegraphics[width=1.0\textwidth]{img/consp_ex.jpg}
\end{figure}
\end{minipage}\hfill\pause
\begin{minipage}[t]{0.45\textwidth}
What we \textbf{want}:\medskip
{\scriptsize
\begin{verbatim}
{
   "form": "23.1",
   "origin": "Italy",
   "decoration": "none",
   "occurs": "uncommon"
},
{
   "form": "23.2",
   "origin": "Italy, not Padana",
   "occurs": "Mediterranean region;
              North-Italy"
}
\end{verbatim}
}
\end{minipage}\pause\medskip

\begin{minipage}[t]{0.45\textwidth}
\begin{center}
\alert{\textbf{UNSTRUCTURED DATA}}
\end{center}
\end{minipage}\hfill
\begin{minipage}[t]{0.45\textwidth}
\begin{center}
\alert{\textbf{STRUCTURED DATA}}
\end{center}
\end{minipage}
\end{frame}

\section{Text Mining: Theory}%\printSectionYes

\begin{frame}{Classification}
\begin{block}{Definition: Text Mining}\vspace{.5em}
\textbf{Text Mining} is a \alert{general term} covering several different ideas\pause, e.\,g.:
\begin{itemize}[<+->]
\item Information retrieval
\item Statistical analysis
\item \textbf{Information extraction}
\item\dots
\end{itemize}
\end{block}
\end{frame}

\begin{frame}{Why underestimation is bad}
\begin{center}
\begin{figure}
\includegraphics[width=\textwidth]{img/xkcd.png}
\caption{I can relate to this. [Source: xkcd.com/1831]}
\end{figure}
\end{center}
\end{frame}

\begin{frame}{Information Extraction}
\begin{block}{Definition: Information Extraction (IE)}\vspace{.5em}
\enquote{\textit{[IE] is the task of automatically extracting structured information from unstructured [\dots] documents.}}
\end{block}\pause\bigskip

\large{Some other facts about Information Extraction:}\normalsize
\begin{itemize}
\item Computer scientists have a hard time with IE (for over \alert{30 years} now!)\pause
\item IE is \textbf{\alert{really super difficult}} and \alert{\textbf{often inaccurate}}.
\end{itemize}
\end{frame}

\begin{frame}{Sorry!}
\begin{large}
\begin{alertblock}{\large DISCLAIMER}\vspace{.5em}
We neither \emph{can} nor \emph{do} provide a perfect solution or perfect results.

Furthermore, this project is still work in progress.
\end{alertblock}

\end{large}
\end{frame}

\begin{frame}{Five Steps}
\begin{enumerate}
\item Tokenisation and Sentence splitting
\item Lemmatisation
\item Part-of-speech-tagging (POS)
\item Named entity recognition (NER)
\item Relation Extraction
\end{enumerate}
\end{frame}

\begin{frame}{POS-Tagging}
\begin{figure}
\tikzstyle{block} = [rectangle, draw, fill=blue!10, rounded corners, text centered, minimum width=1.5em,font=\footnotesize\ttfamily, node distance=2em]
\begin{tikzpicture}[node distance = .5em, font=\bfseries\large,baseline, text height=1.5ex,text depth=.25ex,highlight/.style={text=orange}]
\node (a) at (0,0) {The};
\node[right=of a,onslide={<2-> highlight}] (b) {quick};
\node[right=of b,onslide={<2-> highlight}] (c) {brown};
\node[right=of c,onslide={<2-> highlight}] (d) {fox};
\node[right=of d] (e) {jump};
\node[right=of e] (f) {over};
\node[right=of f] (g) {the};
\node[right=of g,onslide={<2-> highlight}] (h) {lazy};
\node[right=of h,onslide={<2-> highlight}] (i) {dog};
\node[right=of i] (j) {.};

\node[block,fill=Fuchsia!30,below of=a] {DT};
\node[block,fill=yellow!30,below of=b] {JJ};
\node[block,fill=yellow!30,below of=c] {JJ};
\node[block,fill=RoyalBlue!30,below of=d] {NN};
\node[block,fill=green!30,below of=e] {VBD};
\node[block,fill=orange!30,below of=f] {IN};
\node[block,fill=Fuchsia!30,below of=g] {DT};
\node[block,fill=yellow!30,below of=h] {JJ};
\node[block,fill=RoyalBlue!30,below of=i] {NN};
\node[block,fill=gray!30,below of=j] {.};

\node at (0,-1.5) {};
\node[above of=e,gray,node distance=1.3em,font=\mdseries] {\footnotesize jumps}; 
\end{tikzpicture}
\caption{POS-tagging examples after lemmatisation.}
\end{figure}
\end{frame}

\begin{frame}{POS-Tagging}
\begin{figure}
\tikzstyle{block} = [rectangle, draw, fill=blue!10, rounded corners, text centered, minimum width=1.5em,font=\footnotesize\ttfamily, node distance=2em]
\begin{tikzpicture}[node distance = .2em, font=\bfseries,baseline, text height=1.5ex,text depth=.25ex,highlight/.style={text=orange}]
\node[onslide={<2> highlight}] (a) at (0,0) {Form};
\node[right=of a,onslide={<2> highlight}] (b) {20};
\node[right=of b] (c) {occur};
\node[right=of c] (f) {from};
\node[right=of f] (g) {the};
\node[right=of g,onslide={<2> highlight}] (h) {Augustan};
\node[right=of h] (i) {until};
\node[right=of i] (j) {the};
\node[right=of j,onslide={<2> highlight}] (k) {late};
\node[right=of k,onslide={<2> highlight}] (l) {Tiberian};
\node[right=of l,onslide={<2> highlight}] (m) {period};
\node[right=of m] (n) {.};

\node[block,fill=RoyalBlue!30,below of=a] {NN};
\node[block,fill=Fuchsia!30,below of=b] {CD};
\node[block,fill=green!30,below of=c] {VBZ};
\node[block,fill=orange!30,below of=f] {IN};
\node[block,fill=Fuchsia!30,below of=g] {DT};
\node[block,fill=RoyalBlue!30,below of=h] {NNP};
\node[block,fill=orange!30,below of=i] {IN};
\node[block,fill=Fuchsia!30,below of=j] {DT};
\node[block,fill=yellow!30,below of=k] {JJ};
\node[block,fill=yellow!30,below of=l] {JJ};
\node[block,fill=RoyalBlue!30,below of=m] {NN};
\node[block,fill=gray!30,below of=n] {.};

\node at (0,-1.5) {};
\node[above of=c,gray,node distance=1.2em,font=\mdseries] {\footnotesize occurs};
\end{tikzpicture}
\caption{POS-tagging examples after lemmatisation.}
\end{figure}
\end{frame}

\begin{frame}{Relation Extraction}
\begin{center}
\begin{tabularx}{\textwidth}{CCC}
\toprule
\textbf{Subject}&\textbf{Relation}&\textbf{Object}\\\midrule
quick brown fox&jump over&lazy dog\\
Form 20&occur&Augustan\\
Form 20&occur&late Tiberian period\\
\bottomrule
\end{tabularx}
\end{center}
\end{frame}

\section{Text Mining: Practical}

\begin{frame}{Tools}
presenting different tools here
\end{frame}

\begin{frame}[fragile]{Adapting the NER}
Stanford CoreNLP only recognises 8 entities types:

\begin{center}\ttfamily
\begin{tabular}{lll}
PERSON&&DATE\\
ORGANIZATION&&TIME\\
LOCATION&&MONEY\\
PERCENT&&MISC
\end{tabular}
\end{center}

So we have to add the custom type \texttt{FORM}. Adjusting \texttt{DATE} also necessary.
\end{frame}

\begin{frame}[fragile]{\texttt{iepy} Active Learning Core}
nuthin yet
\end{frame}

\section{Multilingualism}

\maketitle

\end{document}
