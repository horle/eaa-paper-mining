\documentclass[xcolor=x11names, aspectratio=169]{beamer}
\usepackage[british]{babel}
\usepackage{amsmath}
\usepackage{amssymb}
\usepackage{amsfonts}
\usepackage{mathpazo}
\usepackage{enumerate}
\usepackage{array,booktabs}
\usepackage{tikz}
\usepackage{mathdots}
\usepackage{verbatim}
\usepackage{multirow}
\usepackage{tabularx}
\usetikzlibrary{matrix,backgrounds,patterns,arrows,decorations.markings,shapes}
\usepackage{caption}

\usetheme[titleformat title=regular,titleformat frame=regular,titleformat section=allcaps,numbering=fraction]{metropolis}

\author[F.\ Kußmaul \&\ T.\ Evans]{{\Large Felix Kußmaul}\hfill\raisebox{-.9em}{\includegraphics[width=4cm]{img/uco.jpg}}\\[1.2em]{\Large Dr Tim Evans}\hfill\raisebox{-.7em}{\includegraphics[width=4cm]{img/york.jpg}}}
\title[Mining Paper Catalogues]{\Large Mining Paper Catalogues}
\subtitle{\normalsize A Multilingual Solution to Reduce Verbose Fields to Consistent Terminology}
\institute[Cologne, York]{European Association of Archaeologists 2017}
\date[31 August 2017]{\ \\[.5em]31 August 2017}

\input{config.tex}
\definecolor{morange}{HTML}{FF8200}
 
\begin{document}

\maketitle

\section{Motivation}

\begin{frame}{Data Source}
\begin{center}
\begin{figure}
\includegraphics[width=.875\paperwidth]{img/consp.jpg}
\caption{Sample from \emph{Conspectus} catalog.}
\end{figure}
\end{center}
\end{frame}

\begin{frame}[fragile]{}
\begin{minipage}[t]{0.45\textwidth}
What we \textbf{have}:\medskip

\begin{figure}
\includegraphics[width=1.0\textwidth]{img/consp_ex.jpg}
\end{figure}
\end{minipage}\hfill\pause
\begin{minipage}[t]{0.45\textwidth}
What we \textbf{want}:\medskip
{\scriptsize
\begin{verbatim}
{
   "form": "23.1",
   "origin": "Italy",
   "decoration": "none",
   "occurs": "uncommon"
},
{
   "form": "23.2",
   "origin": "Italy, not Padana",
   "occurs": "Mediterranean region;
              North-Italy"
}
\end{verbatim}
}
\end{minipage}\pause\medskip

\begin{minipage}[t]{0.45\textwidth}
\begin{center}
\alert{\textbf{UNSTRUCTURED DATA}}
\end{center}
\end{minipage}\hfill
\begin{minipage}[t]{0.45\textwidth}
\begin{center}
\alert{\textbf{STRUCTURED DATA}}
\end{center}
\end{minipage}

\end{frame}

\begin{frame}
\begin{tabbing}
\qquad\textbf{Problem:} \= Running texts contain a lot of \alert{irrelevant information}.\\[.5em]

\> Information scientists call them \textbf{redundant}.
\end{tabbing}
\end{frame}

{ % all template changes are local to this group.
	\setbeamercolor{background canvas}{bg=black}
    \setbeamertemplate{navigation symbols}{}
    \begin{frame}[plain]
        \begin{tikzpicture}[remember picture,overlay]
            \node[at=(current page.center)] {
                \includegraphics[height=\paperheight]{img/death.jpg}
            };
        \end{tikzpicture}
     \end{frame}
}

\section{Text Mining: Theory}%\printSectionYes

\begin{frame}{Classification}
\textbf{Text Mining} is a \alert{general term} covering several different ideas, e.\,g.:\pause

\begin{itemize}[<+->]
\item Information retrieval
\item Cluster analysis
\item Statistical (lexical) analysis
\item \textbf{Information extraction}
\item\dots
\end{itemize}
\end{frame}

\begin{frame}{Information Extraction}
\textbf{Information extraction} (IE) is the task of automatically extracting structured information from unstructured [\dots] documents.

\end{frame}

\begin{frame}{Relation Extraction}
The most common approach is to look for certain \alert{relation triples} in the text.
\end{frame}

\section{Text Mining: Practical}

\begin{frame}[fragile]{Stanford CoreNLP: Output}
\begin{quote}
Subform 20.1 is never common, but has a long history from the Augustan period until the late Tiberian or early Claudian period.
\end{quote}
results in
\begin{verbatim}
have history until: Subform 20.1, tiberian
have history until: Subform 20.1, late tiberian
have history until: Subform, tiberian
have history until: Subform, late tiberian
\end{verbatim}
\end{frame}

\section{Multilingualism}

\maketitle

\end{document}
